\chapter{I sistemi quantistici} \label{ch:mq}

\section{Il teorema di risoluzione spettrale e la formulazione
generale della Meccanica Quantistica}\sectionmark{Formulazione 
generale della M. Q.}

\subsection{Teorema di risoluzione spettrale}

Ogni operatore autoaggiunto 
\`{e} esprimibile nella forma\cite{VON_NEUMANN, CALDIROLA_SPETTRALE}
\begin{equation}
        \hat{A}=\int_{-\infty}^{\infty}a\, d\hat{E}(a),
                \label{eq:risoluz_spettrale}
\end{equation}
 dove $\hat{E}$ indica una misura a valori di proiezione definita
sui boreliani $B$ dell'asse reale, che soddisfa le condizioni
\begin{align*}
\hat{E}(\mathbb{R}) & =\mathbf{1},\\
        \hat{E}(B_{1}\cap B_{2}) & =\hat{E}(B_{1})\hat{E}(B_{2})\,.
\end{align*}

\subsection{I postulati della Meccanica Quantistica}

Nella \emph{descrizione\footnote{
  \`E opportuno distinguere \emph{descrizione} da \emph{rappresentazione}:
  il formalismo da cui prende le mosse il presente lavoro prescinde,
  come vedremo, dalla particolare \emph{rappresentazione}.
}di Schr\"odinger} le proprietà quantistiche
di una particella possono ricavarsi dai seguenti 
postulati\cite{CALDIROLA_GEN}:

\begin{enumerate}
  \renewcommand{\theenumi}{\roman{enumi}}
  \item{%
    Ad ogni sistema fisico è associato un opportuno spazio di Hilbert,
    $\mathcal{H}$. Ogni stato del sistema può essere rappresentato
    mediante un elemento $\psi$ di $\mathcal{H}$ con norma uguale a $1$,
    detto \emph{vettore di stato}, che contiene tutte le possibili 
    informazioni sul sistema.

    L'evoluzione temporale di detto vettore è regolata da una equazione
    della forma
    \begin{equation}
      i\hbar\frac{d\psi}{dt} = \hat{H}\psi  \label{eq:schrod}
    \end{equation}
    dove $\hat{H}$ è un opportuno operatore autoaggiunto detto
    \emph{operatore di Hamilton} o \emph{hamiltoniana} del sistema e
    l'equazione prende il nome di \emph{equazione di Schr\"odinger}.
  }
  \item{
    Ad ogni grandezza osservabile $A$ corrisponde un operatore
    autoaggiunto $\hat{A}$ nello spazio $\mathcal{H}$, il cui
    spettro discreto $\sigma_d$ e continuo $\sigma_c$ costituiscono
    l'insieme dei possibili risultati di una misura di $A$.

    La probabilità che una misura fornisca un risultato nell'insieme
    $\mathcal{A}$ è data da
    \begin{equation}
      P_{\mathcal{A}} = \int_{\mathcal{A}} \vectornorm{d\hat{E}(a) \psi}^2,
      \label{eq:probab}
    \end{equation}
    dove $\hat{E}$ è la misura a valori di proiezione associata
    ad $\hat{A}$ mediante la \eqref{eq:risoluz_spettrale}.
  }
  \item{
    Se da una misura risulta che il valore della grandezza $A$ 
    appartiene all'insieme $\mathcal{A}$, e con i simboli
    $\psi_i$ e $\psi_f$ si indicano rispettivamente il vettore di stato
    prima e dopo l'osservazione, si ha, a meno di normalizzazioni,
    \[
      \psi_f = \hat{E}(\mathcal{A}) \psi_i .
    \]

    In altre parole, \emph{la misura agisce sullo spazio degli stati come
    un \emph{proiettore} sull'autospazio 
    relativo ai risultati ottenuti}.
  }
  \item{
    Ad una particella nello spazio fisico tridimensionale sono associati
    gli operatori
    \begin{align}
      \vec{\hat{r}} &=  (\hat{x},   \hat{y},    \hat{z}), 
        \label{eq:op_posizione} \\
      \vec{\hat{p}} &=  (\hat{p}_x, \hat{p}_y,  \hat{p}_z),
        \label{eq:op_momento}
    \end{align}
    che formano in $\mathcal{H}$ un insieme irriducibile\footnote{
      Cioè non esiste alcun sottospazio proprio di $\mathcal{H}$ 
      invariante sia per $\mathbf{\hat{r}}$ che per $\mathbf{\hat{p}}$.
    }, soddisfano le regole di commutazione
    \begin{align}
      [\hat{r}_i, \hat{r}_j] &= [\hat{p}_i, \hat{p}_j] = 0, 
      	\label{eq:commutaz_qq_pp} \\
      [\hat{r}_i, \hat{p}_j] &= i\hbar \, \delta_{ij}, 
      	\label{eq:commutaz_qp}
    \end{align}
    e corrispondono alle coordinate cartesiane e ai loro momenti coniugati.
  }
  \item{
    Gli operatori corrispondenti alle altre osservabili\footnote{
      Stiamo considerando grandezze che hanno un analogo classico. 
      Lo \emph{spin}, ad esempio, è un'altra osservabile fondamentale,
      considerare la quale richiederebbe qualche ulteriore precisazione.
    }    
    sono \emph{funzioni} degli operatori fondamentali
    \eqref{eq:op_posizione} e \eqref{eq:op_momento}.
  }
  \item{
    L'operatore $\hat{H}$ che compare nella \eqref{eq:schrod} è
    della forma\footnote{
      In presenza di campo magnetico, per una particella di
      carica $e$ (senza spin) la \eqref{eq:hamiltoniana} diventa
      \[
        \hat{H} = \frac {(\vec{\hat{p}}-e\vec{\hat{A}})^2} {2m} 
        + eV(\vec{\hat{r}}) + U(\vec{\hat{r}})
      \]
    dove $\vec{\hat{A}}$ corrisponde al \emph{potenziale vettore} del campo
    elettromagnetico, $V$ al potenziale scalare, e si è posto $c=1$.
    }
    \begin{equation}
      \hat{H} = \frac{\vec{\hat{p}}^2}{2m} + U(  \vec{\hat{r}}    ) . 
      \label{eq:hamiltoniana}
    \end{equation}
  }
\end{enumerate}

I postulati suddetti si riferiscono a sistemi ad una particella
ma sono facilmente generalizzabili al caso di $N$ particelle. In particolare
la \eqref{eq:hamiltoniana} diventa 
$ \hat{H} = \frac{1}{2m} \sum_{k=1}^{N} \vec{\hat{p}_{k}}^2  + 
U(  \vec{\hat{r}_{1},\dots,\hat{r}_{N}}    ) $, mentre le relazioni
di commutazione \eqref{eq:commutaz_qq_pp} e \eqref{eq:commutaz_qp} 
diventano 
$[\hat{r}_i^{(k)}, \hat{r}_j^{(l)}] = [\hat{p}_i^{(k)}, \hat{p}_j^{(l)}] = 0\,$
 e
$\,[\hat{r}_i^{(k)}, \hat{p}_j^{(l)}] = i\hbar \, \delta_{ij}\delta_{kl}$.
Nel seguito, tuttavia, faremo sempre riferimento per semplicità a sistemi 
costituiti da una singola particella. 

\section{La rappresentazione nello spazio delle posizioni}

Come spazio degli stati possiamo scegliere 
$\mathcal{L}^2(\mathbb{R}^3)$ e come operatori fondamentali gli operatori
di moltiplicazione $(x, y, z)$ e gli operatori differenziali
$\frac{\hbar}{i}(\partial_x, \partial_y, \partial_z)$ a rappresentare
rispettivamente posizione e momento lineare. \`E immediato
verificare che le relazioni di commutazione \eqref{eq:commutaz_qq_pp} 
e \eqref{eq:commutaz_qp}
sono soddisfatte.

In questa rappresentazione, la \eqref{eq:schrod} si esprime:
\begin{equation}
  i\hbar\frac{\partial\Psi(x, y, z; t)}{\partial t} = 
  \left[-\frac{\hbar^2}{2m}\nabla^2 + U(x,y,z)\right]\Psi(x,y,z;t) ,
  \label{eq:schrod-q}
\end{equation}
mentre la \eqref{eq:probab}, riferita agli operatori di
posizione, si scrive:
\[
  P_X(t) = \int_X |\Psi(x,y,z; t)|^2\,dxdydz
\]
mostrando che $|\Psi(x,y,z;t)|^2$ ha il notevole significato di 
\emph{densità di probabilità} di trovare la particella in una
data posizione al tempo $t$. 

\section{Evoluzione temporale}

Se $\Psi$ è un'autofunzione dell'\emph{hamiltoniana}
\[
  H = -\frac{\hbar^2}{2m}\nabla^2 + U 
\]
relativa all'autovalore $\epsilon$, la \eqref{eq:schrod-q} diventa
\[
  i\hbar\frac{\partial\Psi}{\partial t} = \epsilon\Psi
\]
e la sua soluzione generale (noto lo stato iniziale $\Psi(t_0)$) 
sarà del tipo
\[
  \Psi(t) = \Psi(t_0) e^{-\frac{i\epsilon(t-t_0)}{\hbar}}.
\]

Si può dimostrare che, come conseguenza della \eqref{eq:risoluz_spettrale}, 
una generica funzione di $\mathcal{L}^2(\mathbb{R})$ è esprimibile
come sovrapposizione lineare di autofunzioni dell'hamiltoniana:
\[
  \Psi = \sum_{n} c_n \varphi_n + 
  \int_{\sigma_c} d\epsilon\,f(\epsilon)\varphi_{\epsilon}
\]
a patto di considerare anche funzioni $\varphi_{\epsilon}$ non a quadrato
sommabile ma che, integrate mediante una opportuna funzione $f$,
diano come risultato una funzione di $\mathcal{L}^2$ (\emph{autofunzioni
improprie}).

Per la linearità della \eqref{eq:schrod-q}, l'evoluzione temporale
della $\Psi$ sarà
\begin{equation}
  \Psi(t) = \sum_{n} c_n \Psi_n(t_0) e^{-\frac{i\epsilon_n(t-t_0)}{\hbar}} + 
  \int_{\sigma_c} d\epsilon\,f(\epsilon)\Psi_{\epsilon}(t_0)
  e^{-\frac{i\epsilon(t-t_0)}{\hbar}}, \label{eq:evoluz_autofunzioni}
\end{equation}
che, definendo opportunamente l'esponenziale di un operatore hermitiano,
si può esprimere come
\[
  \Psi(t) = U(t,t_0)\Psi(t_0) = e^{-\frac{i\hat{H}(t-t_0)}{\hbar}} \Psi(t_0) ,
\]
dove $U(t,t_0)$ è detto \emph{operatore di evoluzione temporale}, e
l'ultima equazione vale a prescindere dalla rappresentazione.

In conclusione, per conoscere l'evoluzione temporale di un sistema
quantistico descritto dalla hamiltoniana $H$ una strada da seguire
consiste nel risolvere l'equazione agli autovalori $H\psi=\epsilon\psi$
(\emph{equazione di Schr\"odinger indipendente dal tempo}), 
per poi esprimere la funzione d'onda all'istante iniziale come
combinazione lineare di autofunzioni, la cui evoluzione è data
dalla \eqref{eq:evoluz_autofunzioni}.

\clearpage\section{Barriera di potenziale unidimensionale ed effetto tunnel}\sectionmark{Barriera di potenziale ed effetto tunnel}

Nella prima simulazione consideriamo una particella in una dimensione, 
di massa $m$, 
soggetta al potenziale:
\[
U(x)=\begin{cases}
	V,	& \text{per }x \in (x_1, x_2) \\
	0,	& \text{altrove } 
\end{cases}
\]
Lo spettro dell'energia è puramente continuo e comprende tutti i valori
maggiori di $0$. Per un analogo sistema classico l'attraversamento della
barriera è possibile solo per $E>V$, mentre in Meccanica Quantistica
esiste una probabilità non nulla che ciò accada anche per $0<E<V$. Tale
fenomeno va sotto il nome di \emph{effetto tunnel}.
Per darne una dimostrazione dobbiamo 
innanzi tutto individuare una schiera di autofunzioni (improprie)
di energia $E \in\,]0, V[ $.

Possiamo risolvere l'equazione di Schr\"{o}dinger indipendente dal tempo,
\[
\left[-\frac{\hbar^2}{2m}\frac{d^2}{dx^2} + U(x)\right]\psi(x) = E\psi(x) ,
\]
separatamente nelle tre regioni $x<x_1$, $x\in(x_1,x_2)$ e $x>x_2$ e
ottenere
\begin{equation}
\psi_E(x) = \begin{cases}
	A_1 e^{ik_{E}x} + A_2 e^{-ik_{E}x}	&\text{ per } x<x_1\\
	B_1 e^{\chi_{E} x} + B_2 e^{-\chi_{E} x}&\text{ per } x\in(x_1, x_2)\\
	C_1 e^{ik_{E}x} + C_2 e^{-ik_{E}x}      &\text{ per } x>x_2
\end{cases},     \label{psi nelle3zone}
\end{equation} 
dove si è posto
\begin{align*}
k_{E} &= \frac{\sqrt{2mE}}{\hbar}, & \chi_{E} &= \frac{\sqrt{2m(V-E)}}{\hbar}.
\end{align*}

Resta da imporre che $\psi$ appartenga al dominio di autoaggiuntezza 
dell'operatore hamiltoniano $-\frac{\hbar^2}{2m}\frac{d^2}{dx^2} + U$.
Si può dimostrare che ciò equivale a richiedere la continuità della 
funzione con la sua derivata, ovunque e, in modo particolare, nei
punti $x_1$ e $x_2$, dove è verificata solo per alcuni valori dei 
coefficienti $A_1$, $A_2$, $B_1$, $B_2$, $C_1$ e $C_2$.

Sappiamo che una funzione d'onda moltiplicata per un arbitrario
fattore complesso rappresenta lo stesso stato. Dunque possiamo
scegliere liberamente il valore di uno dei coefficienti: nel nostro 
caso imporremo $A_1~=~1$.

Vogliamo costruire un pacchetto d'onda che abbia un ``analogo
classico''. Scegliamo, nella nostra simulazione, di rappresentare
una particella che proviene dalla sinistra della barriera. Questo
ci porta ad imporre $C_2=0$, in quanto una sovrapposizione di funzioni
del tipo $e^{-ikx}$ nella regione $x>x_2$ corrisponderebbe ad una 
particella \emph{proveniente da destra} (sono tutte autofunzioni
relative ad autovalori \emph{negativi} del momento lineare).

Abbiamo così ridotto a quattro il numero delle incognite. La richiesta
di continuità di $\psi$ e $\psi'$ in $x_1$ e $x_2$ equivale
ad altrettante equazioni lineari:

\begin{equation}\begin{aligned}
e^{ik_{E}x_1} + A_2 e^{-ik_{E}x_1} &= B_1 e^{\chi_{E}x_1} + B_2 e^{-\chi_{E}x_1} \\
B_1 e^{\chi_{E} x_2} + B_2 e^{-\chi_{E} x_2} &= C_1 e^{ik_{E}x_2} \\
ik_{E} e^{ik_{E}x_1} -ik_{E} A_2 e^{-ik_{E}x_1} &=B_1\chi_{E} e^{\chi_{E} x_1}-B_2 \chi_{E} e^{-\chi_{E} x_1}\\
B_1\chi_{E} e^{\chi_{E} x_2} -B_2 \chi_{E} e^{-\chi_{E} x_2} &= i C_1 k_{E} e^{ik_{E}x_2} .
\end{aligned} \label{sistema_lineare_coeff}  \end{equation} 




\clearpage\sectionmark{Buca bidimensionale}\section{Buca di potenziale bidimensionale a pareti infinite}\sectionmark{Buca bidimensionale}

Per \emph{buca quadrata} s'intende in generale il sistema costituito da una
particella in due dimensioni soggetta al potenziale
\[
U(x,y)=\begin{cases}
        0,      & \text{per }(x,y) \in (-a, a)\times(-a, a) \\
        V,      & \text{altrove }
\end{cases}. \label{eq:potenziale_bucaquadratafinita}
\]

La funzione d'onda deve appartenere al dominio di autoaggiuntezza 
dell'hamiltoniana. Ciò equivale alla
richiesta di continuità della stessa e delle sue derivate spaziali.
Si può dimostrare che, al limite per $V\rightarrow\infty$, 
le condizioni al contorno si riducono a:
\[
\psi(x,y) = 0, \text{ per } (x, y) \in 
	\mathbb{R}^2 - \left(-a,a\right)\times\left(-a,a\right)  .
\label{eq:contorno_buca2dinf}
\]
In termini intuitivi, questa condizione descrive una particella costretta
nel quadrato $(-a, a)\times(-a, a)$ da un potenziale che valga zero
all'interno e sia ``infinito'' al di fuori di tale porzione di piano.


Per semplicità di calcolo scegliamo unità di misura tali da soddisfare 
la relazione $\frac{\hbar^2}{2m}=1$; si pone inoltre $a=\frac{\pi}{2}$.
In questo modo l'operatore hamiltoniano ha la seguente espressione:
\begin{equation*}\label{eq:H_buca}
    H=H_{1}+H_{2}\text{,}
      \quad\text{con}
      \quad H_{1}=-\frac{\partial^{2}}{\partial x^{2}}
      \quad\text{e}\quad H_{2}=-\frac{\partial^{2}}{\partial y^{2}} 
\end{equation*}
e le sue autofunzioni sono tutte del tipo:
\begin{equation}
\psi_{mn}(x,y) = \phi_m(x)\phi_n(y) , \label{eq:autofunz_buca}
\end{equation}
con
\begin{equation}\phi_k(\xi)=\begin{cases}
  \sqrt{\frac{2}{\pi}}\cos(k\xi), & \text{ per } k \text{ dispari } \\
  \sqrt{\frac{2}{\pi}}\sin(k\xi), & \text{ per } k \text{ pari }
\end{cases}. \label{eq:phi} \end{equation}

I corrispondenti autovalori dell'energia\footnote{
  Almeno quando $m\not=n$, si tratta chiaramente di livelli
  degeneri: è sufficiente scambiare gli indici per avere due stati
  $\psi_{mn}$ e $\psi_{nm}$ linearmente indipendenti ma relativi alla
  stessa energia, il che riflette la simmetria del sistema.
} sono dati da $E_{mn}=m^2+n^2$. 
Infatti:
\begin{multline*} H\psi_{mn}(x,y) = 
\left(-\frac{\partial^2}{\partial x^2}-\frac{\partial^2}{\partial y^2}\right)
\phi_m(x)\phi_n(y) = \\
m^2\phi_m(x)\phi_n(y) + n^2\phi_m(x)\phi_n(y) = (m^2+n^2)\psi_{mn}(x,y).
\end{multline*}

Ciascuna delle autofunzioni ha la seguente evoluzione temporale:
\begin{equation}
\Psi_{mn}(x,y;t) = \psi_{mn}(x,y) e^{-iE_{mn}t} 
= \phi_{m}(x) \phi_{n}(y) e^{-i(m^2+n^2)t} . \label{eq:evol_autofunzioni}
\end{equation} 

Per la simulazione prenderemo in considerazione alcune semplici combinazioni
lineari (finite) di autostati dell'energia, cosa resa possibile dallo
spettro discreto dell'operatore hamiltoniano.


