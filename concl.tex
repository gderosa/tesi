\chapter{Conclusioni}

Abbiamo mostrato come sia possibile superare, con 
opportuni accorgimenti, le difficoltà nella rappresentazione grafica
delle funzioni a valori complessi; rappresentazione 
importantissima in Meccanica Quantistica giacché, come è noto, 
la funzione d'onda \emph{deve essere complessa} se si vuole che 
l’equazione di Schr\"odinger~---~equazione differenziale del 
primo ordine rispetto al tempo~---~abbia 
soluzioni ondulatorie \cite{MERZBACHER_WAVES}. Solo gli
autostati dell'energia sono descrivibili da funzioni reali: sono stati
stazionar\^{i} e 
la loro evoluzione temporale è banale in quanto l'argomento complesso
cambia globalmente con un'unica frequenza $\omega=E/\hbar$. 

Più complicata, per certi versi più interessante, e necessariamente 
descritta da funzioni complesse, è invece l'evoluzione di stati 
non stazionar\^{i}, che abbiamo
simulato considerando sovrapposizioni discrete o continue
di autofunzioni relative a energie diverse. L’evoluzione di tali stati
rende evidenti sia la propagazione ondosa che le ``traiettorie'' 
di pacchetti localizzati.

Il presente lavoro è suscettibile ovviamente di sviluppi ulteriori,
come si è già visto in \S\ref{subsec:buca2d_sviluppi}. 

L'evoluzione più
naturale è la simulazione di sistemi a tre dimensioni, disegnando
ad esempio le superfici di livello di $|\Psi|^2$ o le linee di forza della
corrente di probabilità $\vec j = \frac{\hbar}{2mi}\left(\Psi^* \vec
\nabla \Psi - \Psi \vec \nabla \Psi^*\right)$, mentre l’argomento
complesso della $\Psi$ può essere rappresentato con una tavolozza di colori
in modo analogo a quanto visto per la buca bidimensionale.

Vale la pena di osservare, infine, che proprio l'uso del colore
costituisce un problema di accessibilità per le
persone con limitazioni visive, il che può motivarci a sviluppare
rappresentazioni alternative oltre che ad approfondire la vasta
problematica legata alla comunicazione dei risultati scientifici.
