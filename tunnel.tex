% included in
%\chapter{Le simulazioni}\label{ch:simula}

\section{Barriera di potenziale unidimensionale ed effetto tunnel}
\sectionmark{Barriera di potenziale ed effetto tunnel}

\subsection{Calcolo simbolico con \emph{Maxima}}

\emph{Maxima}\cite{MAXIMA} è un \emph{Computer Algebra System} a sorgente 
aperto\cite{OPENSOURCE} con il quale calcoleremo, per via esatta e
simbolica, i parametri della \eqref{psi nelle3zone} e, per via
approssimata e numerica, una sovrapposizione di autofunzioni per
costruire un pacchetto d'onda e studiarne l'evoluzione temporale. 

\emph{Maxima} è un programma \emph{a riga di comando}, per il quale
sono disponibili alcuni \emph{frontend} grafici, come 
\emph{wxMaxima}\cite{wxMAXIMA}.

Di seguito illustreremo i passaggi concettualmente più importanti
dello script utilizzato, riportato in \ref{sec:tunnel.mxm}, cominciando
dalla parte \emph{simbolica}.

Esprimiamo innanzi tutto in codice \emph{Maxima} la 
\eqref{psi nelle3zone} e la sua derivata:
\begin{lstlisting}

/* 
 * general solutions for potential barrier (0<E<V) 
 */
psi1(x) := exp(%i*x*sqrt(2*m*E)/hbar) + A2*exp(-%i*x*sqrt(2*m*E)/hbar);
psi2(x) := C1*exp(x*sqrt((V-E)*m)/hbar) + C2*exp(-x*sqrt((V-E)*m)/hbar);
psi3(x) := B1*exp(%i*x*sqrt(2*m*E)/hbar);
/* 
 * and their derivatives 
 */
Dpsi1(x) := diff(psi1(x),x);
Dpsi2(x) := diff(psi2(x),x);
Dpsi3(x) := diff(psi3(x),x);
\end{lstlisting}

Impostiamo e risolviamo il sistema \eqref{sistema_lineare_coeff}:
\begin{lstlisting}

/* 
 * Regularity condition on wavefunction and derivative 
 */
reg12 : psi1(x1)=psi2(x1);
reg23 : psi2(x2)=psi3(x2);
Dreg12 : Dpsi1(x1)=Dpsi2(x1);
Dreg23 : Dpsi2(x2)=Dpsi3(x2);
/*
 * Simbolically solve the liner equations in A2,C1,C2,B1
 */
linsolve([reg12,reg23,Dreg12,Dreg23],[A2,C1,C2,B1]);
\end{lstlisting}

\`E importante che \emph{Maxima} risolva il sistema simbolicamente 
poiché le soluzioni ottenute sono funzioni di $E$ e di altri
parametri del problema che dobbiamo ancora fissare. Ad esempio, 
alla richiesta\footnote{ Del tutto ridondante se si esegue il codice
di \S\ref{sec:tunnel.mxm} in modalità non interattiva, ma utile a scopo
esplicativo o di \emph{debug}.}:
\begin{lstlisting}

tex(psi1(x));

\end{lstlisting}
si ottiene il seguente output\footnote{ Sfruttiamo la possibilità,
per \emph{Maxima}, di produrre codice {\TeX} che ci consente,
con piccole modifiche, di formattarne i risultati nel presente documento.}:


{\scriptsize
$$ %% copia/incolla da Maxima + modifiche per una migliore formattazione
e^{{{\sqrt{2}\,i\,\sqrt{m}\,x\,\sqrt{E}}\over{{\it \hbar}}}} - 
$$
$$
 \left({e ^ {- {{\sqrt{2}\,i\,\sqrt{m}\,x\,\sqrt{E}}\over{{\it \hbar}}} }\,
 \left(\left(-V-E\right)\,e^{{{2\,\sqrt{m}\,{\it x_2}\,\sqrt{V-E}
 }\over{{\it \hbar}}}+{{2\,\sqrt{2}\,i\,\sqrt{m}\,{\it x_1}\,\sqrt{E}
 }\over{{\it \hbar}}}}+\left(V+E\right)\,e^{{{2\,\sqrt{m}\,{\it x_1}\,
 \sqrt{V-E}}\over{{\it \hbar}}}+{{2\,\sqrt{2}\,i\,\sqrt{m}\,{\it x_1}
 \,\sqrt{E}}\over{{\it \hbar}}}}\right)}\right) /
$$
$$
 \left({\left(2\,\sqrt{2}\,i\,
 \sqrt{E}\,\sqrt{V-E}-V+3\,E\right)\,e^{{{2\,\sqrt{m}\,{\it x_2}\,
 \sqrt{V-E}}\over{{\it \hbar}}}}+\left(2\,\sqrt{2}\,i\,\sqrt{E}\,
 \sqrt{V-E}+V-3\,E\right)\,e^{{{2\,\sqrt{m}\,{\it x_1}\,\sqrt{V-E}
 }\over{{\it \hbar}}}}}\right) .
$$
}

Come si vede, questa è una espressione complicata,
generale e non semplificata, che possiamo considerare come un
elemento intermedio e interno alla elaborazione complessiva.

Con ovvio significato dei simboli possiamo ora scrivere
\[
\psi_E(x) = \begin{cases}
	e^{ik_{E}x} + A_2(E) e^{-ik_{E}x}	&\text{ per } x<x_1\\
	B_1(E) e^{\chi_{E}x} + B_2(E) e^{-\chi_{E}x} &\text{ per } x\in(x_1, x_2)\\
	C_1(E) e^{ik_{E}x}			&\text{ per } x>x_2
\end{cases},
\]
per considerare l'evoluzione temporale di ciascuna autofunzione,
\[
\Psi_E(x;t) = \psi_E(x) e^{-\frac{iEt}{\hbar}} ,
\]
in termini di codice \emph{Maxima}:
\begin{lstlisting}

/* time evolution of each eigenfunction */
Epsi1(x,E,t) := psi1(x)*exp(-%i*E*t/hbar);
Epsi2(x,E,t) := psi2(x)*exp(-%i*E*t/hbar);
Epsi3(x,E,t) := psi3(x)*exp(-%i*E*t/hbar);
\end{lstlisting}

\subsection{Calcolo numerico con \emph{Maxima}}\label{sec:numerica}

A questo punto, completata la parte simbolica della  
elaborazione, possiamo fissare i parametri del problema. Una scelta
opportuna può rendere più o meno evidenti taluni effetti fisici
nel risultato  finale.
\begin{lstlisting}

m       : 2;
hbar    : 0.1;
V       : 1;
x1      : 5.0;
x2      : 5.1;
\end{lstlisting}

Vogliamo ora costruire un pacchetto d'onda localizzato,
sovrapponendo autofunzioni relative a un intervallo di energie
strettamente incluso in $(0,V)$: 
\begin{equation}
\Psi(x;t) = \int_{E_0-\Delta E}^{E_0+\Delta E} f(E) \Psi_E(x;t)\,dE 
\label{eq:superposition}
\end{equation}
dove, per la nostra simulazione si è scelto:
\begin{lstlisting}

E0      : 0.5;
sigmaE  : 0.1;
deltaE  : 1.5 * sigmaE;
/* 
 * Make a superposition of eigenfunctions modulated by f(E). 
 * A gaussian is a good choice to obtain a well localized packet.
 */
f(E) := exp(-(E-E0)^2/sigmaE^2);
\end{lstlisting}

Non resta che valutare numericamente la \eqref{eq:superposition}
in un reticolo discreto di valori per $x$, $t$, e con un opportuno passo.
\begin{lstlisting}

/* boundaries for plotting ... */
x0                : -2;
x3                : 10;
xstep             : 0.0141;
... 
ti                : 1.30;
tf                : 14.8;
tstep             : 0.03750;
Emin              : E0 - deltaE;
Emax              : E0 + deltaE;
...

\end{lstlisting}
Il seguente frammento si riferisce all'integrazione numerica
nel tratto $x<x_1$ (codice analogo è eseguito per le altre due
regioni):
\begin{lstlisting}

...
  /* The superposition (wavepacket) is constructed  */
  for x: x0 step xstep thru x1 do (
    result_re : quad_qags(  /*numerically integrate real... */
      realpart(f(E)*Epsi1(x,E,t)),
      E,
      Emin,
      Emax,
      epsrel,
      limitq
    ),
    result_im : quad_qags( /*...and imaginary part*/
      imagpart(f(E)*Epsi1(x,E,t)),
      E,
      Emin,
      Emax,
      epsrel,
      limitq
    ), 
    print(x, "= x < x1; t =", t, " results:", result_re, result_im),
    printf(stream,format,t,x,result_re[1],result_im[1])
  ),
...
\end{lstlisting}

Abbiamo omesso alcuni dettagli tecnici per i quali si rimanda
al codice completo in \S\ref{sec:tunnel.mxm} oltre che,
naturalmente, alla documentazione di \emph{Maxima}\cite{MAXIMA}. 

I risultati del calcolo sono scritti nella forma 
\begin{align*}
t	&& x 	&& \Re(\Psi(x;t)) 	&& \Im(\Psi(x;t)) 
\end{align*}
in un file
di testo che possa esser letto da \emph{Gnuplot}\cite{GNUPLOT},
per poi realizzare immagini e animazioni.
\begin{lstlisting}

...
stream: openw(output_file);
format: "~f ~f ~f ~f~%"; /* float */
...
  /* The superposition (wavepacket) is constructed  */
  for x: x0 step xstep thru x1 do (
...
    printf(stream,format,t,x,result_re[1],result_im[1])
...
\end{lstlisting}


\subsection{Rappresentazione grafica \label{sec:tunnel-graph}}
Lo script \emph{Perl} riportato in \S\ref{sec:tunnel-plot.pl} 
integra diverse componenti software,
realizza una sequenza di grafici con \emph{Gnuplot}\cite{GNUPLOT} 
e ne ricava file di animazioni mediante diversi codificatori video.
\`E stato sviluppato in ambiente Unix, ma con qualche accorgimento
è possibile il \emph{porting} verso altri sistemi operativi.

Per conoscerne la sintassi, eseguiamolo senza argomenti:
\begin{lstlisting}

$ ./tunnel-plot.pl
Usage: ./tunnel-plot.pl {file.dat} [ size-X size-Y ]

\end{lstlisting}
Bisogna fornire dunque il file di dati prodotto da \emph{Maxima}
ed eventualmente le dimensioni desiderate per i grafici di funzione. 
I parametri scelti in \S\ref{sec:numerica} possono portare
a una esecuzione che può durare diverse ore, ma ci consentono di avere
dati sufficienti a produrre immagini ad una definizione
paragonabile alla massima disponibile sullo schermo di un moderno
personal computer.

Per prima cosa, lo script suddivide il file in ingresso in parti
più piccole, ognuna delle quali contiene dati che si riferiscono
ad uno stesso istante di tempo: in pratica, ciascuna di esse servirà
a produrre un singolo ``fotogramma''.
\begin{lstlisting}

...
open(FH,"<$infile");
#
while(<FH>) {
    chomp;
    if (m/^\s*(\d*\.\d+)/) {
        $line = $_;
        $t = $1;
        $n = sprintf("%07.4f",$t);
        $file = "$datadir/$n$ext";
        unless ($handles{"$n"}) {
            message("splitting: $n$ext ...  \r");
            open($handles{"$n"},">$file") or die ("Couldn't open $file: $!\n");
        }
        $fh = $handles{"$n"};
        print $fh "$line\n";
    }
}
...
\end{lstlisting}

Successivamente viene invocata l'esecuzione di 
\emph{Gnuplot}\cite{GNUPLOT} per produrre i grafici tridimensionali:
\begin{lstlisting}

...
foreach $n (sort(keys(%handles))) {
    $fh = $handles{$n};
    close($fh);
    message("plotting: $n.png ...   \r");
    print GNUPLOT "set output \"$imgdir/$n.png\"\n";
    print GNUPLOT "splot \"$datadir/$n$ext\" using 2:3:4 w lines, ".
    " \"$datadir/$n$ext\" using 2:(0):($smscale*(\$3**2 + \$4**2)) w lines,".
    " $x1, u, v w lines, $x2, u, v w lines\n";
}
...

\end{lstlisting}
nei quali l'asse $x$ è riferito alla variabile indipendente, mentre
gli assi $y$ e $z$ rappresentano rispettivamente, per ogni istante $t$,
la parte reale e immaginaria di $\Psi(x;t)$. 

Alcuni esemp\^i sono riportati nelle Figure 
\ref{fig:tunnel_first}-\ref{fig:tunnel_last} dove, per raffronto,
è riportata anche la usuale rappresentazione del modulo quadro (nel
piano $xz$, in verde). In Fig. \ref{fig:tunnel_first}
è mostrato l'avvicinamento della particella alla barriera, con evidente
analogia al caso classico. In Fig. \ref{fig:impact} si vede come la 
forma d'onda sia alterata dall'impatto con la stessa. 
Nelle Figure \ref{fig:tunnel_effect} e \ref{fig:tunnel_last} 
è mostrato l'\emph{effetto tunnel}. In un analogo sistema classico 
la particella tornerebbe indietro, mentre in questo sistema quantistico
la funzione d'onda sembra ``dividersi in due''. 
Dal punto di vista statistico ciò significa che esiste una probabilità 
non nulla di trovare la particella oltre la barriera.

\begin{figure}\begin{center}
  \subfloat[]{
    \includegraphicshp{img/tunnel/01_0750.png}
  } \\
  \subfloat[]{
    \includegraphicshp{img/tunnel/04_6375.png}
  } 
  \caption{La particella si dirige verso la barriera.}
  \label{fig:tunnel_first}
\end{center}\end{figure}
\begin{figure}\begin{center}
  %\ContinuedFloat%
  \subfloat[]{
    \includegraphicshp{img/tunnel/05_5000.png}
  } \\
  \subfloat[]{
    \includegraphicshp{img/tunnel/06_1750.png}
  }
  \caption{Impatto con la barriera.}\label{fig:impact}
\end{center}\end{figure}
\begin{figure}\begin{center}
  %\ContinuedFloat%
  \subfloat[]{
    \includegraphicshp{img/tunnel/07_0750.png}
  } \\
  \subfloat[]{
    \includegraphicshp{img/tunnel/07_9000.png}
  }
  \caption{Effetto tunnel.}\label{fig:tunnel_effect}
\end{center}\end{figure}
\begin{figure}\begin{center}
  %\ContinuedFloat%
  \subfloat[]{
    \includegraphicshp{img/tunnel/09_1375.png}
  } \\
  \subfloat[]{
    \includegraphicshp{img/tunnel/10_6000.png}
  }
  \caption{Vi è una probabilità che la particella ``sia passata attraverso''.}
  \label{fig:tunnel_last}
\end{center}\end{figure}              

Con i parametri scelti disponiamo di 369 immagini che, alla frequenza
tipica di venticinque \textit{frames} al secondo, ci consentono di
produrre una animazione della durata di $14.76$ secondi.
\begin{lstlisting}

...
    system("$mencoder $uri -o $lossless -ovc lavc -lavcopts vcodec=ffv1");
...
# codice per la conversione ad altri formati...
...
\end{lstlisting}

Nello script in \S\ref{sec:tunnel.mxm} è presente una sezione con
i seguenti altri parametri, \emph{commentati}:
\begin{lstlisting}

/*
x0                : 3.0;
x3                : 6.5;
ti                : 5;
tf                : 12;
tstep             : 0.007;
...
*/

\end{lstlisting}
che ci consentirebbero, se sostituiti ai parametri cui ci siamo riferiti
sinora, di ottenere una animazione molto più dettagliata, nel tempo,
di quanto accade in prossimità della barriera.

Tutte le animazioni, i grafici, i file di dati, il codice 
\emph{Perl} e \emph{Maxima} sono disponibili alla
pagina web \url{http://people.na.infn.it/~pq-qp/pages/simulations.html},
parte del \emph{portale} didattico ``$PQ-QP$''\cite{PQ-QP}.

