
\chapter*{Introduzione} \label{introduzione}
\addcontentsline{toc}{chapter}{Introduzione}
\chaptermark{Introduzione}


La Meccanica Quantistica ha avuto una gran mole di successi
sperimentali ed è, con i suoi sviluppi, l'unica teoria
in grado di descrivere i fenomeni legati alla natura intima
della materia. \`E però fondata su un 
formalismo matematico sofisticato e in genere piuttosto lontano
dall'intuizione. Esistono diversi modi, più o meno generali,
di formulare i concetti fondamentali della teoria; il più noto
è certamente basato sulla descrizione di Schr\"odinger e sulla
rappresentazione nello spazio delle posizioni, ovvero
sull'intuizione di L. V. De Broglie circa la duplice natura,
corpuscolare e ondulatoria, delle particelle materiali. 

La maggior parte delle simulazioni al calcolatore, a scopo didattico
o di ricerca, si basa sulla \emph{Meccanica Ondulatoria}, proprio
perché è la via più efficace per fornire una visione immediata 
di come la teoria ``funzioni'' e sia in grado di prevedere
i fenomeni naturali. Il significato della \emph{funzione d'onda}
come probabilità di trovare una particella in una regione dello spazio
ci motiva a investigare, per via simulata oltre che sperimentale,
le eventuali analogie o le profonde differenze fra un \emph{pacchetto}
d'onda localizzato in una regione finita ed una particella classica 
--- pensiamo agli stati coerenti di un oscillatore armonico piuttosto che
all'effetto tunnel. Per non dire delle domande che ci possiamo
porre quando confrontiamo le \emph{onde di materia} con altri oggetti
di natura classicamente ondulatoria come 
la radiazione elettromagnetica --- si pensi alla diffrazione di un fascio di 
elettroni da parte di un reticolo cristallino piuttosto che alle relazioni
di dispersione non lineari col conseguente \emph{slargamento} del 
pacchetto associato ad una particella libera.

In questo lavoro di tesi è descritto lo studio effettuato per
simulare l'evoluzione temporale di due particolari
sistemi: una particella in una dimensione in
presenza di una barriera di potenziale ed una buca bidimensionale
a pareti infinite. Considerare sistemi ad una o due dimensioni è utile 
non solo perché il calcolo è più semplice
e ci si può concentrare sui concetti fondamentali, ma anche perché,
di fatto, molti sistemi naturali o riproducibili in laboratorio 
hanno caratteristiche tali da poter
essere studiati senza il reale bisogno di considerare tutte
le dimensioni dello spazio fisico. Sistemi siffatti rendono 
più agevole la rappresentazione grafica delle funzioni che li
descrivono e dunque le simulazioni che realizzeremo.

\`E tipico rappresentare una funzione d'onda in una dimensione
mediante il grafico del quadrato del modulo, ma in questo modo
si perde una parte importante dell'informazione sul sistema. 
Se è vero che una funzione d'onda normalizzata è significativa a 
meno di un fattore di fase \emph{costante}, è anche
vero che, per esempio, $\psi(x)$ ed $e^{i\alpha x}\psi(x)$
descrivono particelle aventi la stessa distribuzione di probabilità 
per la posizione ma una
%(ben) 
diversa distribuzione del momento lineare! (In termini
matematici, basti applicare la proprietà di traslazione della
trasformata di Fourier). Per questi motivi si è scelto di rappresentare 
l'evoluzione nel tempo 
della funzione \emph{complessa}. Per un sistema unidimensionale
si possono utilizzare grafici a tre dimensioni, uno dei cui assi sarà
associato alla variabile indipendente mentre gli altri due corrisponderanno
alla parte reale e alla parte immaginaria della funzione. In questo
modo il grafico di $e^{ikx}$ avrebbe la forma di un'elica, ma si possono 
fare numerosi altri esemp\^{i}. 

Nella prima simulazione è stato considerato un pacchetto d'onda con una 
energia (o meglio con una
distribuzione statistica di valori dell'energia) strettamente inferiore
al livello della barriera: in un analogo sistema classico l'attraversamento
di quest'ultima è impossibile. La soluzione delle equazioni quantistiche
mostra invece che esiste una probabilità di trovare la particella
\emph{oltre} la barriera. Fra i successi della meccanica quantistica
possiamo annoverare l'interpretazione del decadimento $\alpha$ 
proprio in termini di questo \emph{effetto tunnel}. Nella
nostra schematizzazione i parametri del sistema sono stati scelti
in modo tale da rendere un simile effetto piuttosto evidente.

La seconda simulazione riguarda un sistema bidimensionale. %
%Sull'asse
%$z$ rappresenteremo nel modo usuale $|\psi(x,y)|^2$, e useremo
%come quarta dimensione del grafico una scala di colori a rappresentare
%tutta e sola l'informazione che manca: l'argomento complesso. 
Una volta usati gli assi $x$ e $y$ per le variabili indipendenti,
sull'asse $z$ può essere rappresentato, nel modo usuale, $|\psi(x,y)|^2$,
mentre l'argomento complesso può essere reso da una scala di colori.
Si è combinato linearmente un numero finito e piccolo di autostati 
dell'energia, ma per 
il resto sono state scelte combinazioni del tutto arbitrarie. Solo in alcuni
casi si è ottenuto un comportamento che potremmo definire
semiclassico, e in tali casi si è potuto vedere che, cambiando solo 
l'argomento complesso di uno dei coefficienti (lasciando
inalterata, dunque, la distribuzione statistica dell'energia), si 
hanno ``traiettorie'' del tutto diverse. 

Prima di illustrare, nel Cap. \ref{ch:simula}, le simulazioni e 
la loro implementazione, dedicheremo il Cap. \ref{ch:mq}
a un breve richiamo dei postulati fondamentali della meccanica
quantistica, facendo riferimento ad alcuni concetti di teoria spettrale. 
Seguiranno considerazioni relative
alla rappresentazione nello spazio delle configurazioni che ci 
consentiranno di esprimere, nel linguaggio della meccanica
ondulatoria, l'equazione agli autovalori dell'hamiltoniana e 
l'evoluzione temporale delle autofunzioni, ovvero gli strumenti 
indispensabili per impostare il seguito di questo lavoro.
%\pagestyle{headings}
\vspace{5pt} \\  %% dirty....
\textbf{
	Tutte le animazioni, i grafici, i file di dati e il codice
	sorgente sono disponibili alla
	pagina web \url{http://people.na.infn.it/~pq-qp/pages/simulations.html},
	parte del \emph{portale} didattico ``$PQ-QP$''\cite{PQ-QP}.
}

