\chapter*{Ringraziamenti} \label{ringraziamenti}
\addcontentsline{toc}{chapter}{Ringraziamenti}

Grazie ai miei relatori, per il dettaglio e per
la visione d'insieme. \`{E} stata un'esperienza istruttiva e divertente.

Grazie ai ragazzi che sotto la guida di Ofelia Pisanti
e Fedele Lizzi hanno lavorato a 
\emph{PQ-QP}~\cite{PQ-QP}. In modo particolare Deborah Pallotti e
Mariano Barbieri che hanno aperto la strada alle simulazioni descritte
in queste pagine. A Mariano devo soprattutto l'opera paziente 
di \emph{fine tuning} relativa alla scelta ottimale dei parametri per 
l'effetto tunnel. 

Grazie a questa bella comunità di fisici napoletani; a chi ha condiviso 
l’entusiasmo degli iniz\^{i} e a chi questo entusiasmo, d'un tratto, 
l'ha visto e l'ha fatto rinascere.

Grazie in modo speciale ai miei genitori, per l'amore e 
la pazienza infiniti; e a tutta
la mia famiglia per la passione con cui mi segue.

Grazie agli amici, in senso lato e in senso stretto, dentro e fuori 
l'Università. Grazie a Luigi, Rossana, Rosario, Tiziano, 
Chiara, Massimo, Sergio, Vittorio, Michele, Ciro, Luca, Gianluca, 
Antonello, Serena, Lello, Ettore, Francesco, Antonio, Cristina, 
Felicia, Carmine, Anna, Diego, Patrizia, Laura, Nunzia, Tiziana, 
Giampaolo, Nello, Emanuele, Igor,
Antonella, Mario, Guido, Giuseppe, Paolo, Sandra, Claudia, Nico, 
Daniela, Simona, Sabino, Roberto, Antonia, Maria Antonietta, Orazio, Marco; 
ciascuno per un motivo diverso. Grazie agli omonimi, che mi aiutano
ad accorciare la lista e soprattutto grazie a tutti coloro di cui
mi sono dimenticato.

%
%\begin{figure}\begin{center}
%  \includegraphicshp{img/BohrHeisenberg.jpg}
%  \caption{Grazie, ragazzi.}
%  \label{fig:solvay}
%\end{center}\end{figure}

