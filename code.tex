
\chapter{Codice}\label{ch:code}

\section{Barriera di potenziale in una dimensione ed effetto tunnel}%
\sectionmark{Barriera unidimensionale ed effetto tunnel}

\subsection{\texttt{tunnel.mxm}: codice \emph{Maxima} per il calcolo
numerico e simbolico}\label{sec:tunnel.mxm}%
\lstinputlisting[numbers=left,numberstyle=\tiny]{code/tunnel.mxm}

\subsection{\texttt{tunnel-plot.pl}: script per la rappresentazione
grafica e le animazioni}\label{sec:tunnel-plot.pl}%
\lstinputlisting[language=Perl,numbers=left,numberstyle=\tiny]{code/tunnel-plot.pl.fmt}


\section{Buca di potenziale bidimensionale}

\subsection{\texttt{buca2d.anima.pl}: script Perl}\label{sec:buca2d.anima.pl}%
\lstinputlisting[language=Perl,numbers=left,numberstyle=\tiny]{code/buca2d.anima.pl}

\subsection{\texttt{buca2d.preamble.gpi}: impostazioni \emph{Gnuplot}}%
\label{sec:buca2d.preamble.gpi}%
\lstinputlisting[numbers=left,numberstyle=\tiny]{code/buca2d.preamble.gpi}

\subsection{\texttt{buca2d.plot.gpi}: codice \emph{Gnuplot} 
per i grafici}\label{sec:buca2d.plot.gpi}%
\lstinputlisting[numbers=left,numberstyle=\tiny]{code/buca2d.plot.gpi}

\subsection{\texttt{myfunc.gpi}: esempio di definizione
di funzione d'onda nella sintassi di \emph{Gnuplot}}%
\label{sec:myfunc.gpi}%
\lstinputlisting[numbers=left,numberstyle=\tiny]{code/myfunc.gpi}

